\PassOptionsToPackage{unicode=true}{hyperref} % options for packages loaded elsewhere
\PassOptionsToPackage{hyphens}{url}
%
\documentclass[]{ctexart}
\usepackage{lmodern}
\usepackage{amssymb,amsmath}
\usepackage{ifxetex,ifluatex}
\usepackage{fixltx2e} % provides \textsubscript
\ifnum 0\ifxetex 1\fi\ifluatex 1\fi=0 % if pdftex
  \usepackage[T1]{fontenc}
  \usepackage[utf8]{inputenc}
  \usepackage{textcomp} % provides euro and other symbols
\else % if luatex or xelatex
  \usepackage{unicode-math}
  \defaultfontfeatures{Ligatures=TeX,Scale=MatchLowercase}
\fi
% use upquote if available, for straight quotes in verbatim environments
\IfFileExists{upquote.sty}{\usepackage{upquote}}{}
% use microtype if available
\IfFileExists{microtype.sty}{%
\usepackage[]{microtype}
\UseMicrotypeSet[protrusion]{basicmath} % disable protrusion for tt fonts
}{}
\usepackage{geometry}
\geometry{a4paper,scale=0.8}
\usepackage{hyperref}
\hypersetup{
            pdftitle={网络流常见建模总结},
            pdfborder={0 0 0},
            breaklinks=true}
\urlstyle{same}  % don't use monospace font for urls
\usepackage{longtable,booktabs}
% Fix footnotes in tables (requires footnote package)
\IfFileExists{footnote.sty}{\usepackage{footnote}\makesavenoteenv{longtable}}{}
\usepackage{graphicx,grffile}
\makeatletter
\def\maxwidth{\ifdim\Gin@nat@width>\linewidth\linewidth\else\Gin@nat@width\fi}
\def\maxheight{\ifdim\Gin@nat@height>\textheight\textheight\else\Gin@nat@height\fi}
\makeatother
% Scale images if necessary, so that they will not overflow the page
% margins by default, and it is still possible to overwrite the defaults
% using explicit options in \includegraphics[width, height, ...]{}
\setkeys{Gin}{width=\maxwidth,height=\maxheight,keepaspectratio}
\setlength{\emergencystretch}{3em}  % prevent overfull lines
\providecommand{\tightlist}{%
  \setlength{\itemsep}{0pt}\setlength{\parskip}{0pt}}
\setcounter{secnumdepth}{0}
% Redefines (sub)paragraphs to behave more like sections
\ifx\paragraph\undefined\else
\let\oldparagraph\paragraph
\renewcommand{\paragraph}[1]{\oldparagraph{#1}\mbox{}}
\fi
\ifx\subparagraph\undefined\else
\let\oldsubparagraph\subparagraph
\renewcommand{\subparagraph}[1]{\oldsubparagraph{#1}\mbox{}}
\fi

% set default figure placement to htbp
\makeatletter
\def\fps@figure{htbp}
\makeatother


\title{网络流常见建模总结}
\author{panda\_2134}
\date{2018/3/4}

\begin{document}
\maketitle

自己对最近网络流学习的一些整理和理解\ldots{}\ldots{} \\

如果有什么错误,请立即纠正,非常感谢。

\newpage
\tableofcontents
\newpage

\hypertarget{header-n2712}{%
\section{最大流}\label{header-n2712}}

\hypertarget{header-n2713}{%
\subsection{二分图相关}\label{header-n2713}}

\hypertarget{header-n2714}{%
\subsubsection{定义与算法}\label{header-n2714}}

\hypertarget{header-n2715}{%
\paragraph{概念}\label{header-n2715}}

\textbf{没有奇环}的图是二分图。二分图可以分成2个部分,每个部分内没有边。为了方便可以称为左点和右点。

匹配:一组顶点不相交的边集合

完美匹配:每个点都是匹配点的匹配

未盖点:不与任何匹配边邻接的点

交替路:
\textbf{未匹配边}-匹配边-未匹配边-匹配边-未匹配边\ldots{}\ldots{}

增广路:以未匹配边结尾的交替路

增广路定理:对于任意图,图上匹配为最大匹配的充要条件是没有增广路

\hypertarget{header-n2730}{%
\paragraph{KM算法}\label{header-n2730}}

 求二分图\textbf{最大权完美匹配} 。

 给每个节点分配一个顶标。定义满足 \(L_i+L_j \geq w_{i, j}\) 的顶标 \(L\)
是可行顶标,满足 \(L_i+L_j = w_{i, j}\)
的边及其顶点构成了相等子图。我们可以证明,相等子图有完美匹配,则它是原图的最大匹配。证明过程对所有的``最大''/``最小''问题都很有启发性:先证明上界,再碰到上界。由于可行顶标的性质,显然相等子图匹配权值
\(\geq\)
原图任何一个完美匹配。在相等子图中可行顶标式子的``=''取得,于是这是原图的最大权匹配。得证。


KM算法步步都用到了贪心的思想。首先贪心地构造出初始相等子图,不妨令每个左点顶标为出边边权最大值,右点顶标为0。每次先从左点开始进行匈牙利算法,求相等子图的一个匹配。如果这个匹配是完美匹配,那么算法结束,否则我们需要让更多边加入进来,从这个点完成一次增广,再从下一个点开始进行匹配。我们给每个在匈牙利算法中访问了的左点顶标减去一个数
\(d\) ,给访问了的右点顶标加上一个数 \(d\) ,来加入一条边。

 分析可知,如果设左点中访问了的点为 \(\mathbb{X}\) 集,未访问的为
\(\mathbb{X'}\) 集,右点相应为 \(\mathbb{Y, Y'}\)
集,那么,\(\mathbb{X} \rightarrow \mathbb{Y'}\)
一定没有边(否则匈牙利树可以继续生长),
\(\mathbb{X'} \rightarrow \mathbb{Y}\)
的边一定是未匹配边;\(\mathbb{X'} \rightarrow \mathbb{Y'}\)
也一样,不过它和我们这步操作无关。再来分析刚才的 \(d\)
应该设为多少。显然 \(d\) 应该\textbf{贪心地}取
\(\min\{L_u + L_v - w_{u, v} | u \in \mathbb{X}, v \in \mathbb{Y'}\}\)
。只能取这个值,因为如果 \(d\) 更大,对于取得 \(\min\) 的边来说,\(u\)
一端顶标就不再可行;如果 \(d\) 更小,那么就没有新边加入。


加了这条边之后原图有什么变化?对于两端都是已访问节点的边而言,由于两端顶标和不变(\((L_u - d) + (L_v + d) = L_u + L_v\)),它是否在相等子图这点并不会改变。对两端都不是已访问节点的边也是一样。左端在
\(\mathbb{X}\) 右端在 \(\mathbb{Y'}\)
的边中边权最大的会加入相等子图。而左端在 \(\mathbb{X'}\) 右端在
\( \mathbb{Y}\)
的边,虽然可能离开相等子图,但是它们本来就不是匹配边,离开了也没有关系。不断进行这步操作直到可以增广。于是这步操作至少引入了一条匹配边。由于上述贪心,最终求出的一定是最大完美匹配。


注意,这个算法只适用于有完美匹配的情况。没有完美匹配的情况下,如果要求最大权匹配,就得用费用流了。

常用性质:

\begin{itemize}
\item
  \(L_u + L_v \geq w_{u,v}\)
\end{itemize}

\begin{itemize}
\item
  \textbf{最大}权匹配等于\textbf{最小}顶标和
\item
  算法结束的时候 \(\sum L_i\) 最小
\end{itemize}

\hypertarget{header-n2756}{%
\paragraph{König定理}\label{header-n2756}}

\textbf{无权二分图的最大基数匹配等于最小点覆盖。}

 从网络流的角度容易证明。选择某点到点覆盖集合中,则割它与 s/t
相连的边。求最大匹配,可以用最大流。由最大流最小割定理,它们是等价的。

\hypertarget{header-n2761}{%
\subsubsection{应用与建模}\label{header-n2761}}

以下的应用,均是在二分图中的。

\hypertarget{header-n2764}{%
\paragraph{最小点覆盖}\label{header-n2764}}

定义:选出最少的点,覆盖图中所有边。

 不带权的时候,由König定理易得。求最大基数匹配即可。


带权的时候,考虑用不带权的类比。不带权的时候,我们是把最大基数匹配(最大流)转为了最小点覆盖(最小割)。现在点有点权,我们同样考虑用最小割建模。建立超级源点
s ,汇点 t ,s 向左点连容量为点权 的边,右点向 t
连容量为点权的边,原二分图中边的容量为无穷大。显然割一定只和 s, t
有关,割那条边就代表把对应点选入最小点覆盖模型\footnote{最小点覆盖和最小割的对应关系,感性理解是显然的。但是我并不会证明。如果有哪位神犇懂得证明请评论,我非常感谢。
}。

\hypertarget{header-n2771}{%
\paragraph{最大点独立集}\label{header-n2771}}

定义:选出最多的点,使得任意边两端点最多有一个被选中。


这个问题和最小点覆盖互补。先考虑不带权情况。我们把图上的点划分成2个集合。最小点覆盖集关于所有顶点的补集,即为最大点独立集。为什么呢?我们考虑图中每条边。其顶点至少一个属于最小点覆盖集,取补集后,最多一个属于最大点独立集,符合其定义。而最小点覆盖集是最小化集合中点数目,取补集后为最大化点独立集中点数目,与最大点独立集的优化目标一致。带点权的情况的证明是类似的。

 所以说,要求最大点独立集大小,用总点数/总点权减去最小点覆盖的大小即可。

\hypertarget{header-n2778}{%
\paragraph{DAG的最小路径覆盖 / 有向图的最小圈覆盖}\label{header-n2778}}

定义:前者为在\textbf{DAG}上选择\textbf{最少条}点不相交的路径,使得这些路径上含有图上\textbf{所有点}。


后者为在有向图\textbf{最少个点不相交的环},使得环上有图上所有点。(注意可以有自环,而且有些时候自环必不可少)

 先看最小路径覆盖。同样分为不带权和带权两种情况。我们把每个点拆成 2
个,一个为左点,一个为右点。考虑在最小路径覆盖中,除了路径结尾每个点都有\textbf{唯一的后继},也就是说每个点和它的后继点一一匹配。从另一个角度看,每次匹配后继点,对应于把两条路径合并起来,并的次数最多的时候,最终剩下的路径条数也就最少了\footnote{来自\href{http://hzwer.com/1758.html}{hzwer学长}}。要是每个边有代价,并且要求最小化总代价和呢?给匹配边赋边权。在路径之间转移有代价?考虑从
s 直接连边,也就是表示某个点是 s
这个虚拟点的后继节点。(SDOI2010星际竞速)


后者和前者相似,也是``匹配后继''的思想。不过有解的充要条件是对应二分图有完美匹配。(可以用置换来思考:有完美匹配,即可以看作一个置换,而置换一定可以分解为若干循环乘积)

\hypertarget{header-n2787}{%
\paragraph{稳定婚姻问题}\label{header-n2787}}

 用图论的话来说,把边权变成了``双向不同''的。 \(u \rightarrow v\)
边权大,\( v \rightarrow u\) 边权\textbf{不一定}大。求一个匹配使得不存在
\(<u,v> \in \mathbb{E}\) ,其中 \(u\) 已经和 \(b\) 匹配,而 \(v\) 已经和
\(a\) 匹配,且对于 \(u\) 来说 \(b\) 不如 \(v\) , 而且对于 \(v\) 来说
\(a\) 不如 \(u\) 。

 Gales-Shapley
算法:每次男子按照喜爱度大到小依次求婚,女子如果发现当前配偶对自己吸引力不如现在求婚的大,就可以抛开当前配偶与现在求婚的结婚。可以证明最后的解一定稳定。

\hypertarget{header-n2792}{%
\paragraph{建模套路}\label{header-n2792}}

\begin{itemize}
\item
  给出矩阵,看作邻接矩阵,转为二分图来理解 (uvaoj11419、ZJOI矩阵游戏)
\item
  给出矩阵,黑白染色,再把两种颜色的看成二分图(骑士共存问题、清华集训2017无限之环)
\item
  与``阶段''有关的匹配问题,\textbf{按照阶段拆点}建图。有时候阶段对应点很多,要动态开(SCOI修车,NOI美食节)
\item
  二分图完美匹配等价于置换,利用置换循环的关系帮助思考(ZJOI矩阵游戏)
\end{itemize}

\hypertarget{header-n2808}{%
\subsection{最小割}\label{header-n2808}}

最小割:一个 s-t 划分

最大流最小割定理:s-t 最大流 = s-t 最小割


满流边不一定都在最小割中。跑完Dinic后在\textbf{残量网络中DFS/BFS求出最小割}。(考虑Amber神犇论文里面举的典型错误!)

\hypertarget{header-n2815}{%
\paragraph{建模}\label{header-n2815}}

\begin{itemize}
\item
  涉及到集合的划分问题,就想到最小割。(uvaoj1515,ZJOI狼和羊的故事)
\item
  有向图的最大闭合子图。选 u, v
  中的一个就会产生某个代价,但是都选代价不会更高。 s
  向每个非负权值点连边,每个负权值点向 t 连边,求出割 \([S,T]\) 后,
  \(S-\{s\}\) 即最大闭合子图,其权值为 \(\sum w_+ - c[S,T]\)
  。(NOI2009植物大战僵尸:注意虽然没有自环,但是可能连环保护导致无敌,而且无敌点保护的点也无敌)
\item
  无向图的最大密度子图。边和点都带有权值,求一个子图,最大化
\end{itemize}

\[\frac{\sum_{e \in \mathbb{E'}}w_e}{\sum_{v \in \mathbb{V'}}w_v}\]

\begin{itemize}
\item
  显然是分数规划。首先二分答案 \(x\)
  。选了边,相邻点就要选,可以把每条无向边拆成 2
  条有向边。再看最大闭合子图权值是否大于 0 ,从而调整二分的答案。(uvaoj
  Hard Life)(顺便吐槽下这个题目,卡精度,eps开1e-6
  WA,开1e-8连样例都过不了,浪费了我 2 个多小时\ldots{}\ldots{})
\end{itemize}

\hypertarget{header-n2831}{%
\subsection{拆点拆边技巧}\label{header-n2831}}

\begin{itemize}
\item
  节点容量拆成边,转为边容量
\item
  某个东西有``阶段''的划分,每个阶段拆出一个点
\item
  二分图匹配是利用 s 到左点的容量来限制最多匹配 1
  条边,可以类似地进行``三分图匹配''(酒店之王)
\end{itemize}

\hypertarget{header-n2842}{%
\section{费用流}\label{header-n2842}}

\hypertarget{header-n2843}{%
\subsection{算法}\label{header-n2843}}

SPFA-Edmonds-Karp / Primal-Dual

比较如下:(测试用题为洛谷费用流模板)\\

\begin{longtable}[]{@{}lll@{}}
\toprule
Algorithm & Accepted & Time\tabularnewline
\midrule
Dijkstra+Pairing Heap+Primal Dual(O2) & Yes & 820ms\tabularnewline
Dijkstra+std::priority\_queue+Primal Dual(O2) & Yes &
832ms\tabularnewline
Dijkstra+Binary Heap with decrease\_key+Primal Dual(O2) & Yes &
888ms\tabularnewline
Dijkstra+Pairing Heap+Primal Dual & Yes & 1236ms\tabularnewline
Dijkstra+Binary Heap with decrease\_key+Primal Dual & Yes &
1528ms\tabularnewline
SPFA+Primal Dual(O2) & Yes & 1548ms\tabularnewline
SPFA+Edmond Karp & Yes & 1596ms\tabularnewline
SPFA+Primal Dual & Yes & 2184ms\tabularnewline
Dijkstra+std::priority\_queue+Primal Dual & No & 3036ms\tabularnewline
SPFA+SLF+Primal Dual(O2) & No & 3204ms\tabularnewline
SPFA+SLF+Primal Dual & No & 4740ms\tabularnewline
\bottomrule
\end{longtable}

\hypertarget{header-n2897}{%
\subsection{建模}\label{header-n2897}}

\hypertarget{header-n2898}{%
\subsubsection{求k条路径并最短路}\label{header-n2898}}

\begin{itemize}
\item
  求 \(s \rightarrow t\) 的 \(k\) 条路径,总长度最短。每条边容量 1
  ,费用为边权,直接找固定流量的最小费用流。
\end{itemize}

\hypertarget{header-n2903}{%
\subsubsection{最小费用可行流}\label{header-n2903}}

\begin{itemize}
\item
  最小费用循环流,也称最小费用可行流,对每个点都满足流量平衡的条件。消圈算法太慢,我们\textbf{用建图技巧避开输入的负权边}\footnote{参考了\href{https://artofproblemsolving.com/community/c1368h1020435__zkw}{zkw神犇的Blog}}。每个负权边
  \(<u, v>\) 拆成 \(<ss, v>, <v, u>, <u, tt>\)
  三条,容量均同原来的负权边,只有第二条边带上费用,费用为原来的费用的相反数。再找
  \(<ss, tt>\)
  的最小费用流。最后把费用\textbf{加上}原图所有\textbf{负权边权值和各自容量乘积}即为答案。
\end{itemize}

\begin{itemize}
\item
  为什么这么做?这就是``预先流满''的操作。既然 Bellman-Ford
  算法无法处理负权环,我们就让图中的负权边预先流满,同时累加上流满它们的代价(显然为负数)。在流满之后,残量网络里面就没有负权边了。但是某些点,确切地说,那些与原来负权边相关的点就不再满足流量平衡的条件了。仔细分析发现,我们得给``预先流满''的流找个来头。我们不妨认为它们是从
  \(tt\) 流来的,一直流到 \(ss\) 结束。这样的话,除了 \(ss, tt\) ,
  其他点都满足了流量平衡的条件。\(tt \rightarrow ss\)
  流恰好流满了所有的负权边。既然我们已经把负权边流满了,我们再试图从
  \(tt\) 向 \(ss\)
  增广也没有意义了,因为以后的增广费用一定为正。所以我们令\(<ss,  v>, <u, tt>\)
  的容量等于原来负权边的容量,费用为 0 。
\item
  注意:这样求出的流在删掉附加边之后是不满足流量平衡条件的。如何满足?从
  \(ss\) 向 \(tt\) 增广,以消去附加边!增广到与 \(ss, tt\)
  有关的边退出残量网络即可。这样就消掉了之前假定的 \(tt \rightarrow ss\)
  流,所有的流就都是来自图内部了,删掉 \(ss, tt\) 以及与 \(ss, tt\)
  有关的边,图中就是最小费用可行流,对每个节点都满足流量平衡。
  有人也许会问:在 \(ss \rightarrow tt\)
  增广的过程中,会引入负权环吗?这是不可能的。新图中边的费用均非负,如果要增广产生负权环,必定要沿着某个正环增广。而沿着正环增广不仅无益于增大
  \(s-t\)
  流,还会徒增费用。也就是说,只要图中没有负环,增广后也不会有负环。所以这样一定可以求出合法的最小费用可行流。
\end{itemize}

残量网络如图。图上 \textbf{边权为费用,容量均为1} 。
\begin{figure}
 \centering
 \includegraphics{D:/Blog/img/NetworkFlow/NegativeCycle.png}
 \caption{负权边建图}
\end{figure}

\hypertarget{header-n2919}{%
\subsubsection{带负环的最小费用最大流}\label{header-n2919}}

\begin{itemize}
\item
  求 \(s \rightarrow t\)
  最小费用最大流。注意可以有负环,但是负环要有容量限制。先连上
  \(t \rightarrow s\) 的边,容量 \(\infty\) ,费用 0
  。用上面的方法求最小费用可行流。再从 s 往 t 增广(不拆
  \(t \rightarrow s\)
  边,见上下界网络流)。两次增广的费用之和即为总的最小费用。
\end{itemize}

\hypertarget{header-n2924}{%
\subsubsection{费用与流量成下凸函数的最小费用最大流}\label{header-n2924}}

\begin{itemize}
\item
  差分权值后拆边
\item
  e.g.
  \(cost = flow^2 \Rightarrow \begin{cases}cost=flow \\ cost = 3\cdot flow \\ cost = 5 \cdot flow \\ \cdots\end{cases}\)
\end{itemize}

\hypertarget{header-n2932}{%
\subsubsection{注意事项}\label{header-n2932}}

 一般实际使用的时候,不直接建出与 ss, tt
有关的边,而是用一个数组记录到某个点的边总容量(费用相同,均为0),以去除重边。

\hypertarget{header-n2936}{%
\section{上下界网络流}\label{header-n2936}}

其思想与负权费用流建图有所不同。负权费用流建图的``预先流满''后尝试退流,而上下界网络流则是``强制流满''。

\hypertarget{header-n2937}{%
\subsection[上下界可行流]{\texorpdfstring{上下界可行流\footnote{参考了\href{http://www.cnblogs.com/liu-runda/p/6262832.html}{liu\_runda的Blog}}}{上下界可行流}}\label{header-n2937}}

如下图,为一条下界为 \(2\) ,上界为 \(5\) 的弧。

\begin{figure}
\centering
\includegraphics{D:/Blog/img/NetworkFlow/BoundedFlow1.png} 
\caption{上下界可行流}
\end{figure}

我们把下界非 0 的弧拆成必要弧和附加弧。必要弧一定要满流,附加弧不一定。

 如何让必要弧满流?用附加源点。用 Dinic 找出从 ss 到 tt
的最大流,如果所有和 ss, tt 相关的边都满流,则求出了一个可行流。

\hypertarget{header-n2944}{%
\subsection{\texorpdfstring{上下界 \(s-t\)
最大流}{上下界 s-t 最大流}}\label{header-n2944}}

首先连接边 \(<t, s>\) ,容量无穷大。然后找出一个上下界可行流,不拆
\(<t, s>\) 边,直接求解 \(s \rightarrow t\)
最大流即为答案。很多的资料都说要拆掉 \(<t, s>\)
边,但仔细想想就会发现,这是不必要的。直接原封不动找 \(s \rightarrow t\)
最大流就可以了。

下图为一个要求解上下界网络流的残量网络。

\begin{figure}
\centering
\includegraphics{D:/Blog/img/NetworkFlow/BoundedFlow2.png}
\caption{上下界最大流}
\end{figure}

可以看出,最后一次增广刚好撤销了 \(t \rightarrow s\)
边上的流量!于是最大流为2。


由于代表下界的必要弧已经拿出了原来的图,显然不可能增广到流量低于下界,所以一定合法。

\hypertarget{header-n2954}{%
\subsection{\texorpdfstring{上下界 \(s - t\)
最小流}{上下界 s - t 最小流}}\label{header-n2954}}

 这个问题也只在有流量下界的时候有意义,因为流量下界为 0
时,最小流就是零流,没有什么可求的。

 我们考虑流量的反对称性:

\[f(t, s) = -f(s, t)\]

 也就是说 \(t \rightarrow s\) 流量增加,等价于 \(s \rightarrow t\)
流量减少。于是从 t 往 s 增广即可求出最小流。

 有一点需要注意:求最小流的时候\textbf{必须删掉}最后加的 \(<t,s>\)
弧及其反向弧。否则沿着新加的弧增广,最小流是无穷小。

最终答案为可行 \(s-t\) 流减去增广的 \(t-s\)
最大流。前者可以通过检查求完上下界可行流后 \(<t,s>\) 边的流量得知。

 (好像还有一种更简单的方法\footnote{见\href{http://www.cnblogs.com/mlystdcall/p/6734852.html}{\_\_stdcall's
  Blog}}\ldots{}\ldots{}但是没能理解\ldots{}\ldots{})

\hypertarget{header-n2968}{%
\subsection{上下界最小费用可行流}\label{header-n2968}}

同样是拆边法。但是不能允许必要弧退流。于是我们把必要弧和费用流的附加弧合并处理。对于非负费用边,我们把必要弧拉出来,建立弧
\(<ss, v>, <u, tt>\) ,其容量均为下界,但是只有 \(<ss,v>\)
带上与原弧相同的费用,从而避免必要弧重复计费。对于负费用边,见下图中(a),
(b), (c)。

\hypertarget{header-n2971}{%
\subsection{\texorpdfstring{上下界 \(s-t\)
最小费用最大流}{上下界 s-t 最小费用最大流}}\label{header-n2971}}

注意:可以有负环。大体思想就是把负权费用流和上下界费用流合二为一。如下图。(a) - (c) 步骤是连边 \(<t,s>\) ,容量 \(\infty\)
,费用0,把上下界最小费用最大流转为上下界最小费用可行流处理。(d)
步骤是在可行流基础上求最小费用最大流。注意:总的流量等于 (d)
步骤增广的流量,但是费用等于所有负权值的和+两次增广的费用。

\begin{figure} 
\centering
\includegraphics{D:/Blog/img/NetworkFlow/BoundedCostflow.png}
\caption{上下界最小费用最大流}
\end{figure}

\hypertarget{header-n3229}{%
\subsection{\texorpdfstring{上下界 \(s-t\)
最小费用流}{上下界 s-t 最小费用流}}\label{header-n3229}}

同上下界 \(s-t\)
最小费用最大流,不过要把求最小费用最大流改成求最小费用流,即在 \(s-t\)
距离非负时停止增广。

\end{document}
